\documentclass{article}
\usepackage{amsmath}
\usepackage{amsfonts}
\usepackage{amssymb}

\newenvironment{proof}{\paragraph{Proof:}}{\hfill$\square$}
\newtheorem{theorem}{Theorem}
\newtheorem{lemma}[theorem]{Lemma}
\newtheorem{corollary}[theorem]{Corollary}

\newcommand{\R}{\mathbb{R}}
\newcommand{\Q}{\mathbb{Q}}
\newcommand{\N}{\mathbb{N}}

\newcommand{\B}{\mathbb{B}}

\newcommand{\vol}{\text{vol}}

\author{Arthur Chen}
\title{Chapter 7 Lebesgue Measure}
\date{\today}

\begin{document}
\maketitle

\section*{7.2 Outer Measure}

\subsection*{Exercise 7.2.1}

Prove Lemma 7.2.5. Show that outer measure has the following six properties:

\subsubsection*{v - Empty set}

The empty set $\emptyset$ has outer measure $m^*(\emptyset) = 0$.

Let $\epsilon > 0$, and let $B_i$ be an open box with volume less than $\epsilon$. $B_i$ covers the empty set, for all values of $\epsilon$, so $m^*(\emptyset) \leq \epsilon$ for all $\epsilon > 0$. Thus 0 is a lower bound for the volume of the boxes covering $\emptyset$. In fact, it is a least lower bound, because the volume of a box is nonnegative. Thus $0 = \inf\{\vol(B_i): B_i \text{ covers }\emptyset \} = m^*(\emptyset)$.

\subsubsection*{vi - Positivity}

We have $0 \leq m^8(\Omega) \leq + \infty$ for every measurable set $\Omega$.

The outer measure is the infimum of the volume of open boxes that cover a set. The volume of open boxes is nonnegative, meaning that the infimum is $\geq 0$.

\subsubsection*{vii - Monotonicity}

If $A \subset B \subset \R^n$, then $m^*(A) \leq m^*(B)$.

Let $(B_j)$ cover $B$. Then $(B_j)$ covers $A$, and thus the infimum of the volume of boxes that cover $A$ is less than or equal to the infimum of the volume of boxes that cover $B$.

\subsubsection*{x - Countable Sub-additivity}

If $(A_j)_{j \in J}$ are a countable collection of subsets of $\R^n$, then $m^*(\cup_{j \in J} A_j) \leq \sum_{j \in J} m^*(A_j)$.

Fix $\epsilon > 0$. Because $A$ can be covered by a countable number of boxes, the index set can be relabeled as $J = \{j_1, j_2 \dots \}$.

\begin{lemma}
Let $B$ be an outer measurable set. Then for all $\epsilon > 0$, there exists a countable covering of $B$ by open boxes $(A_j)_{j \in J}$ such that $\sum_{j \in J} \vol(A_j) < m^*(B) + \epsilon$.
\begin{proof}
Suppose not. Then there exists an $\epsilon > 0$ such that for all countable coverings of $B$ by open boxes $(A_j)_{j \in J}$, we have $\sum_{j \in J} \vol(A_j) \geq m^*(B) + \epsilon$. Then $m^*(B) + \epsilon$ is a lower bound for the set 

\[
\{\sum_{j \in J} \vol(A_j): (A_j)_{j \in J} \text{ covers } B; J \text { at most countable}\}
\]
which contradicts $m^*(B)$ being the greatest lower bound for this set.
\end{proof}  
\end{lemma}

Using the lemma, for all $j \in N$, we can cover $A_j$ with countable open boxes with volume less than $m^*(A_j) + \frac{\epsilon}{2^j}$. Denoting these boxes $B_{jk}$, the $B_{jk}$ cover $\cup_{j \in J} A_j)$. Because the $A_j$ and $B_k$ are countable, the $B_{jk}$ are countable. Thus

\[
m^*(\cup_{j \in J} A_j) \leq \sum_{j \in \N, k \in \N} \vol(B_{jk})
< \sum_{j \in \N} (m^*(A_j) + \frac{\epsilon}{2^j}) \leq \sum_{j \in \N} (m^*(A_j)) + \epsilon
\]

Because $\epsilon > 0$ was arbitrary, the result is proved.

\subsubsection*{viii - Finite Sub-additivity}

If $(A_j)_{j \in J}$ are a finite collection of subsets of $\R^n$, then $m^*(\cup_{j \in J} A_j) \leq \sum_{j \in J} m^*(A_j)$.

This just follows from x.

\subsubsection*{xiii - Translation Invariance}

If $\Omega$ is a subset of $\R^n$ and $x \in \R^n$, then $m^*(x + \Omega) = m^*(\Omega)$.

Let $(A_j)_{j \in J}$ be a countable cover of $\Omega$. Consider $A_j + x$. It has volume

\[
\vol(A_j + x) = \prod_{i=1}^n ((b_i + x_i) - (a_i + x_i)) = \prod_{i=1}^n (b_i - a_i) = \vol(A_j)
\]

A covering means that for all $p \in \Omega$, there exists $A_j$ such that $p \in A_j$. For all points $p+x \in \Omega + x$, there exists $A_j + x$ such that $p+x \in A_j + x$. Thus, $(A_j + x)_{j \in J}$ cover $\Omega + x$. Since the volume of $A_j$ is unchanged through translation, the infimum of their union remains unchanged, so $m^*(\Omega+x) = m^*(\Omega)$.

\end{document}